%%%%%%%%%%%%%%%%%%%%%%%%%%%%%%%%%%%%%%%%%
\documentclass[11pt,a4paper,sans]{moderncv} % Font sizes: 10, 11, or 12; paper sizes: a4paper, letterpaper, a5paper, legalpaper, executivepaper or landscape; font families: sans or roman

\moderncvstyle{casual} % CV theme - options include: 'casual' (default), 'classic', 'oldstyle' and 'banking'
\moderncvcolor{blue} % CV color - options include: 'blue' (default), 'orange', 'green', 'red', 'purple', 'grey' and 'black'

\usepackage[brazil]{babel}
\usepackage[utf8]{inputenc}
\usepackage[T1]{fontenc}

\usepackage{lipsum} % Used for inserting dummy 'Lorem ipsum' text into the template
\usepackage{multicol}

\usepackage[scale=0.75]{geometry} % Reduce document margins
%\setlength{\hintscolumnwidth}{3cm} % Uncomment to change the width of the dates column
%\setlength{\makecvtitlenamewidth}{10cm} % For the 'classic' style, uncomment to adjust the width of the space allocated to your name

%----------------------------------------------------------------------------------------
%	NAME AND CONTACT INFORMATION SECTION
%----------------------------------------------------------------------------------------

\firstname{Felipe} 
\familyname{Volpato}
% All information in this block is optional, comment out any lines you don't need
\title{IT Analyst | SysAdm | Linux | VMWare | Docker | Java}
\address{R. Saul Brandalise 1531, ap 802}{Videira, SC 89560-290 - Brazil}
\mobile{+5549999011677}
\email{fvolpato@gmail.com}
\photo[70pt][0.6pt]{picture-3} % The first bracket is the picture height, the second is the thickness of the frame around the picture (0pt for no frame)
\quote{"If you get up in the morning and think the future is going to be better, it is a bright day. Otherwise, it's not". Elon Musk}

%----------------------------------------------------------------------------------------

\begin{document}
\thispagestyle{empty}

\makecvtitle % Print the CV title

%----------------------------------------------------------------------------------------
%	EDUCATION SECTION
%----------------------------------------------------------------------------------------

\section{Education}
\cventry{2015--2017}{Master}{Universidade Federal de Santa Catarina -- UFSC}{Florianópolis}{SC}{Master's in Computer Science (Computer Networks)}
\cventry{2011--2013}{Graduate}{Universidade do Oeste de Santa Catarina -- UNOESC}{Joaçaba}{SC}{Specialist in Databases and Business Intelligence}
\cventry{2005--2009}{Bachelor}{Universidade do Estado de Santa Catarina -- UDESC}{Joinville}{SC}{Bachelor's in Computer Science}  

%------------------------------------------------

\subsection{Thesis}

\cventry{2017}{Architecture for autonomous QoS management in SDN environments}{}{}{}{The Software Defined Networks (SDN) approach is an innovative way to provision and deliver Quality of Service (QoS), however it is still lacking in providing context differentiation services. In this research work, an architecture for autonomous QoS management is proposed to provide QoS management with queue prioritization in an SDN environment.}
%----------------------------------------------------------------------------------------
%	AWARDS SECTION
%----------------------------------------------------------------------------------------

\section{About me}

Graduated from the University of the State of Santa Catarina (UDESC) in Computer Science on August 2009 and, in that same year, worked at the company CH Sistemas with headquarters in Joaçaba-SC. Still in 2009, he held a public tender, was approved, and on February 2010, he started as an IT Analyst at the Federal University of Fronteira Sul (UFFS) in Chapecó-SC, where he worked until October 2012. In search of new challenges, in 2012 he moved to the city of Luzerna-SC (close to his hometown), to work at the Federal Institute of Santa Catarina (IFC) new campus, and then assumed the position of Information Technology Coordinator. On August 2015, he started his master's degree in Computer Science at UFSC, ending in 2017. Since then, he has been working on the following fronts: service virtualization -- VMWare, network management and monitoring -- Zabbix, development of secure network authentication -- PFSense, Unifi, openLDAP, Freeradius, deployment of IPv6, security policy management and unix server management -- mainly Debian.

\vspace{1em}

%----------------------------------------------------------------------------------------
%	COMPUTER SKILLS SECTION
%----------------------------------------------------------------------------------------

\section{My practical IT experience}

I have over 10 years of experience in the information technology field. 

I started my professional career working in the development of an ERP built in Delphi and using Firebird back in 2009.

In 2010 I started working for the Brazilian Government at UFFS. During this period, I developed the pilot of an academic enrollment system using PHP and DBMS Postgresql.

I always felt motivated to work in any area, be it software development or operations. So I started to work with infrastructure/operations and network campus management, becoming the University's Infrastructure Coordinator. I worked on the implementation, support and management of several third-party systems such as: Pergamum, OJS, DSpace, Crafty, SIMEC, Ieducar, Moodle, Joomla (institutional portal), Expresso, among others. I also acted in the management of the UFFS network, promoting internal courses to implement network segmentation and configuration of firewalls and campus interconnections, videoconference and VoIP on all university campuses. We worked with the entire infrastructure equipment park of the UFFS Datacenter.

In 2017 I received my Master's degree in Computer Science in the area of Software-defined Networking that required knowledge (mainly) in computer networks and programming (Java).

I have experience in switching and routing solutions for a lot of brands, implementing and testing the most varied network protocols and systems: ARP, DHCP, DNS, MPLS, RSTP, VLANs , 802.1x, LACP, etc. I also worked/work with wireless control solutions. I have a Professional Enterprise Wireless Administrator (UEWA) certification from Ubiquiti received in 2022. I have a lot of experience with FortiNet Firewall and currently working with IPTables and PFSense.

I was responsible more than once for the company's network infrastructure and systems, which made it possible to work with datacenter servers and blades of different models, always in line with virtualization solutions and containers. I also had the opportunity to work and learn from fantastic people and follow their evolution while leading the group.

I'm addicted to up-to-date documentation because I believe that documentation is a very important part of the job and is crucial for the company. I also spend a lot of time monitoring the network and services using and generating scripts for the fantastic Zabbix tool.

I worked a lot with relational databases such as MySQL, PostgreSQL and Oracle and I have a postgraduate degree in Database and Business Intelligence. I also implemented solutions using openLDAP and MSCHAP allowing secure access control to websites and the wireless network (freeRadius).

I'm passionate about the Java programming language, but I also have experience in Shell Scripting, Python, Javascript, Angular and PHP. I worked with GIT, Maven and Gradle with Eclipse and Visual Studio Code IDEs. 

\section{IT Certifications}

\cventry{2022}{Linux Essentials}{Linux Professional Institute - Online LPI000412611 - 8jua8jeylj -- Brasil}{2022}{}{} 

\cventry{2022}{Ubiquiti Enterprise Wireless Admin UEWA}{Ubiquiti Inc. Florianópolis -- Brasil}{2022}{}{} 

\cventry{2019}{Data Cabling System}{Furukawa Electric LatAm. Curitiba -- Brasil}{2019}{}{}

\vspace{4em}

\section{Strong skills}

\cvlistitem{Unix system admin}
\cvlistitem{Virtualization -- VMWare Enterprise}
\cvlistitem{Computer Networking -- firewall, switching, routing, etc.}
\cvlistitem{Open Source Software}
\cvlistitem{Selenium IDE}
\cvlistitem{Latex}
\cvlistitem{Git, Gitlab}

\section{Great skills}

\cvlistitem{Infra as a code: Vagrant, Ansible, Puppet, Docker}
\cvlistitem{Software-defined Networking and QoS}
\cvlistitem{Programming with Java/PHP/Python}
\cvlistitem{Eclipse, Visual Code}
\cvlistitem{Maven, Gradle}
\cvlistitem{Relational Databases: MySQL, PostgreSQL}
\cvlistitem{Amazon AWS}
\cvlistitem{OpenFlow}

\section{Knowledge in}

\cvlistitem{Backend dev. -- Java Spring Framework}
\cvlistitem{Frontend dev. -- HTML,CSS, Bootstrap, Javascript, JQuery, Angular Framework}
\cvlistitem{Typescript}
\cvlistitem{Hibernate}
\cvlistitem{React \& React Native}
\cvlistitem{Javascript \& CSS}
\cvlistitem{iReport}

\section{Technologies that I worked on}

\cvlistitem{Switches: Huawei, D-Link, 3Com, HP, Ubiquiti and Mikrotik}
\cvlistitem{Storages: Dell PowerVault MD3200i, EMC CX4-240}
\cvlistitem{Servers: Dell PowerEdge R710 e R720, HP Proliant DL120 G6, Blade HP C7000, IBM x3500 M2}
\cvlistitem{Firewall: Fortinet UTM, PFSense, Iptables}
\cvlistitem{Wireless: D-Link DWS-3024 Wireless Controller, D-Link DWL-3200AP, Cisco Aironet 1140, Unifi \& Controller: AP, AP-LR, AP-AC-PRO, AP-AC-HD, UNMS, Airfiber 24, NanoStation, etc..}
\cvlistitem{Virtualization infrastructure: VMWare vCenter, Xen, Oracle VirtualBox}
\cvlistitem{Cloud Infrastructure: Amazon AWS, Contabo, Hostgator, GoDaddy, DigitalOcean}
\cvlistitem{Databases: MySQL, MariaDB, MongoDB, Postgresql, Oracle}
\cvlistitem{LDAP: OpenLDAP}
\cvlistitem{Radius: FreeRadius}
\cvlistitem{Email: Roundcube, Expresso, Sendmail, Postfix, Cyrus, SASL}
\cvlistitem{Network Monitoring: Cacti, Nagios and Zabbix}
\cvlistitem{Backup: Bacula, VDP}
\cvlistitem{Documentation: Mediawiki, Dokuwiki}
\cvlistitem{CMS: Wordpress, Joomla, Drupal, Magento}
\cvlistitem{Networking/Security: Fortigate, PFSense, Packetfence, Squid,  FloodLight, GrayLog, RSyslog }
\cvlistitem{DNS: Bind, GoDaddy}
\cvlistitem{Application Services: JBoss, Apache2, Light HTTPd, \textbf{Nginx}}
\cvlistitem{File Management: Samba, Owncloud, FreeNAS}
\cvlistitem{Helpdesk: GLPI, OsTicket, OTRS}

\section{Personal Skills}

\cvlistitem{Self-learning}
\cvlistitem{Teamwork}
\cvlistitem{Creativity}
\cvlistitem{Innovation}
\cvlistitem{Critical thinking}

\section{International Published Articles}

\cventry{2018}{Proactive Autonomic Semantic Engine to User Experience-Aware Service Provision}{IEEE 23rd International Symposium on Computers and Communications (ISCC). Natal -- Brazil}{From June 25th 2018 through June 28th 2018}{}{}
\cventry{2018}{OFQuality: A Quality of Service management module for Software-Defined Networking}{International Journal of Grid and Utility Computing (IJGUC)}{2018}{}{}
\cventry{2017}{An Autonomic QoS management architecture for Software-Defined Networking Environments}{IEEE 22nd International  Symposium on Computers and Communications (ISCC). Heraklion -- Greece}{From July 3rd 2017 through July 6th 2017}{}{}
\cventry{2017}{An Autonomic QoE-Aware Management Architecture for Software-Defined Networking}{IEEE 26th International Conference on Enabling Technologies: Infrastructure for Collaborative Enterprises (WETICE). Poznan -- Poland}{From June 21st 2017 through June 23rd 2017}{}{}
\cventry{2017}{Provisioning and Delivering Sepsis Data Supported by an Enhanced SDN Environment}{IEEE 30th International Symposium on Computer-Based Medical Systems (CBMS). Thessaloniki -- Greece}{From June 22nd 2017 through June 24th 2017}{}{}

\cventry{2016}{A Software-Defined Network Configuration Providing Differentiated QoS to an eHealth Environment}{Proceedings of the 22nd International Conference on Parallel and Distributed Processing Techniques and Applications (PDPTA). Las Vegas -- USA }{From July 25th 2016 through July 28th 2016}{}{}

\section{National Published Articles}

\cventry{2017}{Proposta de um modelo de abstração da camada de gerenciamento de dispositivos com ênfase em QoS e suporte a OpenFlow}{XVII Escola Regional de Alto Desempenho do Estado do Rio Grande do Sul (ERAD-RS). Ijuí -- Brasil}{Data: 05 a 07 de Abril de 2017}{}{}

\cventry{2010}{Proposta de baixo custo para controle de equipamentos elétricos residenciais através de dispositivos móveis}{(UDESC's Bachelor Project)}{Congresso Computer on the Beach 2010. Florianópolis -- Brasil}{From March 19th 2010 through March 21st 2010}{}

%----------------------------------------------------------------------------------------
%	LANGUAGES SECTION
%----------------------------------------------------------------------------------------

\section{Languages}
\cvitemwithcomment{Portuguese}{Native/Proficient}{}
\cvitemwithcomment{Spanish}{Intermediary}{}
\cvitemwithcomment{English}{Advanced}{}

%----------------------------------------------------------------------------------------
%	INTERESTS SECTION
%----------------------------------------------------------------------------------------

%\section{Interesses Particulares}
%\cvlistdoubleitem{Fotografia}{Games}
%\cvlistdoubleitem{Séries de TV}{Serviços de Streaming}
%\cvlistdoubleitem{Apreciar vinhos}{Viajar}
%\cvlistdoubleitem{Bolsa de Valores}{Estudar inglês}

\section{Personal Data}
\cvitem{Birth}{February 20th 1986}
\cvitem{Address}{R. Saul Brandalise, 1531, AP. 802, CEP: 89560-290, Videira -- SC -- Brazil}
\cvitem{Language}{Portuguese, Spanish, English}
\cvitem{Phone}{+5549999011677}
\cvitem{Email}{fvolpato@gmail.com}
\cvitem{Facebook} 
{\includegraphics[width=0.6cm,height=0.4cm]{facebook}http://www.facebook.com/fvolpato2}
\cvitem{Instagram}
{\includegraphics[width=0.6cm,height=0.4cm]{instagram}http://www.instagram.com/fvolpato2}
\cvitem{LinkedIn}{
\includegraphics[width=0.6cm,height=0.4cm]{linkedin}https://br.linkedin.com/in/felipe-volpato-0926853b}
\cvitem{Github}{
\includegraphics[width=0.5cm,height=0.5cm]{github} https://github.com/dcc6fvo/}

\section{}
%\cvitem{}{I hereby declare that the above mentioned information is correct up to my knowledge and I bear the responsibility for the correctness of the above mentioned particular.}
%\cvitem{}{Date:}
%\cvitem{}{Place:}
\cvitemwithcomment{}{}{\huge \color {black!20} Felipe Volpato}

\end{document}